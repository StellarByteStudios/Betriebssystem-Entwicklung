\documentclass[11pt,
a4paper,
oneside,
abstracton,
cleardoublepage=empty,]{scrartcl}


% = = = Imports in der Präambel: = = = %
\usepackage{datetime} % Import für Monat Zahl zu Text
\usepackage{ifthen}   % Für die Anonymisierung
\usepackage{wrapfig}  % Um Bilder direkt in den Text einzubetten

\usepackage[utf8]{inputenc}
\usepackage[ngerman]{babel} % Für die deutsche Sprache
\usepackage{amsmath, amssymb, amsthm} % Einige hilfreiche Mathe-Pakete
\usepackage{tabularx, multirow} % Für Tabellen
\usepackage{graphicx} % Grafiken

\usepackage[automark]{scrlayer-scrpage} % Fußzeile



% Graphendarstellung
\usepackage{tikz}
\usetikzlibrary{intersections, 
  arrows.meta, 
  positioning, 
  shapes.geometric, 
  decorations.pathmorphing,
  calc
}

\usepackage{color} % Für bessere Farbkontrolle
\usepackage{iftex} % Für die PDF Metadaten
\usepackage{hyperref} % Auch für die PDF Metadaten: 'hyperref' immer als letztes Paket laden!










% = = = Sonstige Einstellungen: = = = %
% 
% Metainformationen zur Abgabe
% 

% Anonymisiert? wenn nein auf false umstellen
\newboolean{anonym}
\setboolean{anonym}{false}  % Set the boolean to true or false

% Name des Moduls
\newcommand{\moduleTitle}{Betriebssystem Entwicklung}
% Unterschrift des Übungsblatte
\newcommand{\exerciseTitle}{Userlib als Schnittstelle\\Unabhängige App-Entwicklung}

% Vorname und Name des Studenten
\newcommand{\excerciseAuthor}{Carsten Krollmann}
\newcommand{\excerciseAuthorID}{295 245 7}

% Betreuer des Studenten
\newcommand{\moduleFaculty}{Lehrstuhl für Betriebssysteme}
\newcommand{\supervisor}{Univ-Prof. Dr. Michael Schöttner}

% Aktuelles Datum
\newdateformat{monthyeardate}{%
  \monthname[\THEMONTH], \THEYEAR}
\newcommand{\monthtoday}{\monthyeardate\today}
 % für Erstellung der Titelseite
% = = = Theoremumgebungen: = = = %
%%% Neue Umgebungen können mit "\newtheorem{Name}[Zählung]{Bezeichnung}[Gliederung]" definiert werden.
\newtheorem{theorem}{Theorem}[section] % 'section' sorgt dafür, dass vor dem Zähler noch die Section-Nummer eingefügt wird.
\newtheorem{lemma}[theorem]{Lemma}
\newtheorem{proposition}[theorem]{Proposition}
\newtheorem{definition}[theorem]{Definition}
\newtheorem{bemerkung}[theorem]{Bemerkung}\newtheorem{beispiel}[theorem]{Beispiel}
\newtheorem{annahme}[theorem]{Annahme}
\newtheorem{corollary}[theorem]{Corrollar}
 % Selbsterstellte Commandos laden

% PDF Metadaten
\ifpdf
    \definecolor{brown}{cmyk}{0, 0.81, 1, 0.60}
    \hypersetup{%
      pdftitle={\moduleTitle{}, \exerciseTitle{}},
      pdfsubject={\moduleTitle{} Ausarbeitung},
      pdfauthor={\excerciseAuthor{}},
      colorlinks=false, urlcolor=blue, citecolor=brown,
      bookmarksnumbered=true,
    }
\else   
\fi


% Fußzeile
% Seitenstil aktivieren
\pagestyle{scrheadings}

% Alle Kopf- und Fußzeileninhalte löschen
\clearpairofpagestyles

% Zentrierte Fußzeile
\cfoot{\textit{\excerciseAuthor{}, \excerciseAuthorID{}, \exerciseTitle}}

% Linie oben deaktivieren (offiziell empfohlene Methode)
\KOMAoptions{
  headsepline=false,   % obere Linie AUS
  footsepline=0.4pt    % untere Linie EIN mit 0.4pt Stärke
}









% = = = Beginn des eigentlichen Dokuments: = = = %
\begin{document}

% macht die Erste nummerierung in Römisch, damit der Kontent bei Seite 1 anfängt
\pagenumbering{roman}

% Titelseite einfügen
\pdfbookmark[0]{Titelseite}{Front Page}
\begin{titlepage}
    \centering
    \includegraphics[width=10cm]{fig/Titelseite/Logo_HHU_+Name_horizontal_4c_+Safezone.pdf}\\

    \vfill
    \Huge
    \moduleTitle{}\\*[40pt]
    %\exerciseTitle{}\\*[40pt]
    \normalsize

    
    \huge
    \exerciseTitle{}\\[0.25em]
    \normalsize
    von\\
    \Large
    \ifthenelse{\boolean{anonym}}
    {Anonyme/-r Autor/-in}  % true
    {\excerciseAuthor{}}       % false
    \\

    \vfill


    vorgelegt\\[5mm]
    \supervisor{}\\
    \moduleFaculty{}\\
    Heinrich-Heine-Universität Düsseldorf\\[0.5cm]

    am\\
    %\DTMmonthname{\thesissubmissionmonth} \thesissubmissionyear{}\\[0.5cm]
    \monthtoday{}\\[0.5cm]

\end{titlepage}

% Nächste Seite
\cleardoublepage


% Jetzt normal weiter Seitenzahlen
\pagenumbering{arabic}

% = = Hier kommen die einzelnen Kapitel = = %
\section{Einleitung - Ziele und Motivation}


\begin{frame}{Motivation}
    \begin{Large}
        Apps sind auf Betriebssystem angewiesen \newline
        $\Rightarrow$ starke Abhängigkeit \newline
        \newline
        \onslide<2,3>
        Apps auf unterschiedlichen Systemen \newline
        $\Rightarrow$ enge Schnittstelle \newline
        \onslide<3>
        \newline
        \textcolor{gray}{Beispiel: Syscalls $\Rightarrow$ ein einziger Interrupt}
    \end{Large}
\end{frame}


\begin{frame}{Ziele}
    \begin{Large}
        Ziele:
    \end{Large}
    \vspace{15pt}

    \begin{itemize}
        \item Schnittstelle als Lib
        \item Kernel implementiert Schnittstelle
        \item Apps über Schnittstelle
        \item [] \quad $\Rightarrow$ Systeme austauschbar
    \end{itemize}
\end{frame}


\begin{frame}{Struktur}
    \begin{center}
        \begin{tikzpicture}[node distance=1.2cm and 1cm, every node/.style={minimum width=2.2cm, minimum height=0.9cm, align=center, font=\small}]

        % Farben definieren
        \tikzstyle{app}=[draw, fill=blue!20, ellipse]
        \tikzstyle{lib}=[draw, fill=orange!30, diamond, rounded corners]
        \tikzstyle{kernel}=[draw, fill=green!25, rectangle, aspect=2]

        % Apps
        \node[app] (app1) {App A};
        \node[app, right=of app1] (app2) {App B};
        \node[draw=none, right=of app2] (dots) {\Large$\cdots$};
        \node[app, right=of dots] (app3) {App N};

        % Userlib
        % Userlib – zentriert zwischen app1 und app3
        \node[lib, below=1cm of $(app1)!0.5!(app3)$] (userlib) {Userlib\\(Usermode)};

        % Kernel
        \node[kernel, below left=0.8cm and 1.2cm of userlib] (kernel1) {Kernel 1};
        \node[kernel, below right=0.8cm and 1.2cm of userlib] (kernel2) {Kernel 2};

        % Geschwungene Verbindungen Apps -> Userlib (ungerichtet)
        \draw[->] (app1.south) to[out=-90, in=135] node[midway, above, yshift=-5pt] {API} (userlib.north west);
        \draw[->] (app2.south) to[out=-90, in=90] node[near end, above, yshift=-5pt] {API} (userlib.north);
        \draw[->] (app3.south) to[out=-90, in=45] node[midway, above, yshift=-5pt] {API} (userlib.north east);


        % Userlib → Kernel
        \draw[->] (userlib.south west) to[out=-100, in=90] node[near end, above, yshift=-5pt] {Syscalls} (kernel1.north);
        \draw[->] (userlib.south east) to[out=-80, in=90] node[near end, above, yshift=-5pt] {Syscalls} (kernel2.north);
        \end{tikzpicture}
    \end{center}
\end{frame}


\section{User-Library}

\subsection{Bereits gegeben}
    Im vorherigen Kurs ''Isolation und Schutz in Betriebssystemen'' wurde bereits mit Syscalls gearbeitet. Wir haben dort ein externes Rust-Projekt erstellt, welches die Syscalls abkapselt. Die Apps mussten nur noch diese User-Library importieren. \newline
    Durch diese User-Library war das Grundgerüst gesetzt, jedoch war die User-Library nicht umfangreich genug um verschiedene Apps zu entwickeln.
    

\subsection{Ziele}
    Unser Ziel war es eine rudimentäre Shell zu schreiben, mit welcher man geladene Apps starten und überwachen kann. Gleichzeitig sollten die Apps aber auch auf mehreren Systemen laufen. Daher mussten wir die User-Library so zu modifizieren, dass sie einerseits auf mehreren System läuft. Dazu musste man den Kernel an die Syscall Nummern und deren Spezifikation anpassen. Andererseits sollten die Funktionen der User-Library erweitert werden, sodass diversere Apps möglich sind.

\subsection{Layout}
    \begin{figure}[h] \label{fig:layout-usrlib}
        \centering
        \begin{tikzpicture}[node distance=1.2cm and 1cm, every node/.style={minimum width=2.2cm, minimum height=0.9cm, align=center, font=\small}]

        % Farben definieren
        \tikzstyle{app}=[draw, fill=blue!20, ellipse]
        \tikzstyle{lib}=[draw, fill=orange!30, diamond, rounded corners]
        \tikzstyle{kernel}=[draw, fill=green!25, rectangle, aspect=2]

        % Apps
        \node[app] (app1) {App A};
        \node[app, right=of app1] (app2) {App B};
        \node[draw=none, right=of app2] (dots) {\Large$\cdots$};
        \node[app, right=of dots] (app3) {App N};

        % Userlib
        % Userlib – zentriert zwischen app1 und app3
        \node[lib, below=1cm of $(app1)!0.5!(app3)$] (userlib) {Userlib\\(Usermode)};

        % Kernel
        \node[kernel, below left=0.8cm and 1.2cm of userlib] (kernel1) {Kernel 1};
        \node[kernel, below right=0.8cm and 1.2cm of userlib] (kernel2) {Kernel 2};

        % Geschwungene Verbindungen Apps -> Userlib (ungerichtet)
        \draw[->] (app1.south) to[out=-90, in=135] node[midway, above, yshift=-5pt] {API} (userlib.north west);
        \draw[->] (app2.south) to[out=-90, in=90] node[near end, above, yshift=-5pt] {API} (userlib.north);
        \draw[->] (app3.south) to[out=-90, in=45] node[midway, above, yshift=-5pt] {API} (userlib.north east);


        % Userlib → Kernel
        \draw[->] (userlib.south west) to[out=-100, in=90] node[near end, above, yshift=-5pt] {Syscalls} (kernel1.north);
        \draw[->] (userlib.south east) to[out=-80, in=90] node[near end, above, yshift=-5pt] {Syscalls} (kernel2.north);
        \end{tikzpicture}
    \end{figure}


\subsection{Neue Funktionen}
    Die gemeinsame User-Library \cite{usrlib-repo} wurde um zwei Hauptteile, sowie einige neue Datentypen erweitert

    \subsubsection{Auslagerung unabhängiger Funktionen für Kernel}
        Um Teile des Kernel nicht mehrfach schreiben zu müssen wurden einige Funktionen auch in die User-Library verlagert. Wichtig ist dabei, dass es sich ausschließlich um Funktionen handelt welche keine privilegierten Befehle benutzt. Zu den ausgelagerten Funktionen gehört der \verb|command-parser| die \verb|environment-variablen| und einige Datentypen, sowie mathematische Funktionen wie \verb|cosinus| welche in \verb|#[no_std]| nicht vorhanden sind.


    \subsubsection{Erweiterung der Syscall API}
        Um noch mehr Funktionen des Kernels zu benutzten haben wir weitere Syscalls gebraucht. Diese gehen von kernel-prints über zeichnen von Bitmaps bis zum managen von Threads und Prozessen. \newline
        Ein Ziel war es außerdem auch diese Syscalls mehr abzukapseln, sodass die Apps nur noch Übermethoden verwenden und nicht mehr die Syscalls direkt. So kann man die Komplexität des zusammenbauns verschiedener Syscalls für die Apps verstecken, da die Syscalls lediglich mit \verb|u64| als Datentyp arbeiten. \newline
        Neue Syscalls:
        \begin{itemize}
            \item thread/Prozessmanagement
            \item [] \begin{itemize}
                \item kill, get name/ID
            \end{itemize}
            \item holen von Daten
            \item [] \begin{itemize}
                \item Zeit
                \item Bildschirmgröße
                \item Tastatureingabe
            \end{itemize}
            \item Prints
            \item [] \begin{itemize}
                \item Bildschirm
                \item seriell (kprint)
                \item kerneldaten (VMAs, Apps, threads)
            \end{itemize}
            \item Pixel/Bildzeichnen
            \item Musik abspielen
        \end{itemize}


\section{Apps}



\begin{frame}{Apps Liste}
    \begin{Large}
        Folgende Apps wurden implementiert
    \end{Large}
    \vspace{15pt}

    \begin{itemize}
        \item animation
        \item apps
        \item datetime
        \item echo
        \item kill
        \item mandelbrot
        \item music
        \item pic
        \item threads
        \item uptime
    \end{itemize}
    
\end{frame}



\section{Zusammenfassung}

\subsection{Herausforderungen und Probleme}
    Eine der größten Herausforderung war es abzusprechen welche Syscalls neu hinzugefügt werden sollten, denn jede Änderung an der Syscall-API ist mit größeren Arbeiten im Kernel verbunden. Außerdem wurde das Development der Userlib etwas verlangsamt, da man diese nur über das Internet mit seinem System synchronisiert hat. Das heißt, jede Änderung in der Userlib musste erst hochgeladen werden, ohne dass der neue Code getestet werden konnte.

    \subsubsection{Probleme im eigenen System}
        Es haben sich noch einige Probleme gezeigt, deren Ursprung noch in den vorhergegangenen Modulen liegt. Beispielsweise wurde die \verb|tar|-Datei bei Optimierungen durch den Kompiler während der Laufzeit überschrieben, wodurch der Code der Apps unbrauchbar wurde. \newline
        Außerdem gab es Probleme, wenn man auf die Thread-Queue im Scheduler zugreift. Das war vorher nur beim Thread-Wechsel nötig, aber da wir durch die Threads laufen, um sie auszugeben, wird von verschiedenen Stellen auf die Queue zugegriffen was zu Deadlocks geführt hat.


    \subsubsection{Probleme der Userlib}
        Aufgrund eigener Datentypen musste der Kernel auch auf die Userlib dependen. Das führte zu Dopplungen beim Allocator, Panic Handler und ähnliches. Daher musste die Userlib speziell mit Compile-Flags versehen werden, sodass der Kernel bestimmte Funktionen nicht aus der Userlib bekommt.


\subsection{Lerneffekt}
    Wir mussten als Team Entscheidungen treffen wie die Userlib sich verändern soll. Dabei hat man einerseits die API Änderungen möglichst klein halten müssen. Dadurch hat man oft Syscalls für neue Aufgaben anders verwendet um die API nicht anpassen zu müssen. \newline
    Bei der Entwicklung der Apps hat man viel über Hardware-Nahe Entwicklung gelernt, da durch die \verb|#[no_std]| Umgebung viele Funktionen nicht verfügbar waren, sodass man diese selber Implementieren musste. Das geht von mathematische Funktionen wie \verb|cos, ln| bis hin zum Berechnen und Platzieren von Pixeln.



% Referenzenverzeichnis
\cleardoublepage
\bibliographystyle{abbrv}
\bibliography{sources.bib}

\end{document}
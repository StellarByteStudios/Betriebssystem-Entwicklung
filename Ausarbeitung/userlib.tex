\section{User-Library}

\subsection{Bereits gegeben}
    Im vorherigen Kurs ''Isolation und Schutz in Betriebssystemen'' wurde bereits mit Syscalls gearbeitet. Wir haben dort ein externes Rust-Projekt erstellt, welches die Syscalls abkapselt. Die Apps mussten nur noch diese User-Library importieren. \newline
    Durch diese User-Library war das Grundgerüst gesetzt, jedoch war die User-Library nicht umfangreich genug um verschiedene Apps zu entwickeln.
    

\subsection{Ziele}
    Unser Ziel war es eine rudimentäre Shell zu schreiben, mit welcher man geladene Apps starten und überwachen kann. Gleichzeitig sollten die Apps aber auch auf mehreren Systemen laufen. Daher mussten wir die User-Library so zu modifizieren, dass sie einerseits auf mehreren System läuft. Dazu musste man den Kernel an die Syscall Nummern und deren Spezifikation anpassen. Andererseits sollten die Funktionen der User-Library erweitert werden, sodass diversere Apps möglich sind.

\subsection{Layout}
    \begin{figure}[h] \label{fig:layout-usrlib}
        \centering
        \begin{tikzpicture}[node distance=1.2cm and 1cm, every node/.style={minimum width=2.2cm, minimum height=0.9cm, align=center, font=\small}]

        % Farben definieren
        \tikzstyle{app}=[draw, fill=blue!20, ellipse]
        \tikzstyle{lib}=[draw, fill=orange!30, diamond, rounded corners]
        \tikzstyle{kernel}=[draw, fill=green!25, rectangle, aspect=2]

        % Apps
        \node[app] (app1) {App A};
        \node[app, right=of app1] (app2) {App B};
        \node[draw=none, right=of app2] (dots) {\Large$\cdots$};
        \node[app, right=of dots] (app3) {App N};

        % Userlib
        % Userlib – zentriert zwischen app1 und app3
        \node[lib, below=1cm of $(app1)!0.5!(app3)$] (userlib) {Userlib\\(Usermode)};

        % Kernel
        \node[kernel, below left=0.8cm and 1.2cm of userlib] (kernel1) {Kernel 1};
        \node[kernel, below right=0.8cm and 1.2cm of userlib] (kernel2) {Kernel 2};

        % Geschwungene Verbindungen Apps -> Userlib (ungerichtet)
        \draw[->] (app1.south) to[out=-90, in=135] node[midway, above, yshift=-5pt] {API} (userlib.north west);
        \draw[->] (app2.south) to[out=-90, in=90] node[near end, above, yshift=-5pt] {API} (userlib.north);
        \draw[->] (app3.south) to[out=-90, in=45] node[midway, above, yshift=-5pt] {API} (userlib.north east);


        % Userlib → Kernel
        \draw[->] (userlib.south west) to[out=-100, in=90] node[near end, above, yshift=-5pt] {Syscalls} (kernel1.north);
        \draw[->] (userlib.south east) to[out=-80, in=90] node[near end, above, yshift=-5pt] {Syscalls} (kernel2.north);
        \end{tikzpicture}
    \end{figure}


\subsection{Neue Funktionen}
    Die gemeinsame User-Library \cite{usrlib-repo} wurde um zwei Hauptteile, sowie einige neue Datentypen erweitert

    \subsubsection{Auslagerung unabhängiger Funktionen für Kernel}
        Um Teile des Kernel nicht mehrfach schreiben zu müssen wurden einige Funktionen auch in die User-Library verlagert. Wichtig ist dabei, dass es sich ausschließlich um Funktionen handelt welche keine privilegierten Befehle benutzt. Zu den ausgelagerten Funktionen gehört der \verb|command-parser| die \verb|environment-variablen| und einige Datentypen, sowie mathematische Funktionen wie \verb|cosinus| welche in \verb|#[no_std]| nicht vorhanden sind.


    \subsubsection{Erweiterung der Syscall API}
        Um noch mehr Funktionen des Kernels zu benutzten haben wir weitere Syscalls gebraucht. Diese gehen von kernel-prints über zeichnen von Bitmaps bis zum managen von Threads und Prozessen. \newline
        Ein Ziel war es außerdem auch diese Syscalls mehr abzukapseln, sodass die Apps nur noch Übermethoden verwenden und nicht mehr die Syscalls direkt. So kann man die Komplexität des zusammenbauns verschiedener Syscalls für die Apps verstecken, da die Syscalls lediglich mit \verb|u64| als Datentyp arbeiten. \newline
        Neue Syscalls:
        \begin{itemize}
            \item thread/Prozessmanagement
            \item [] \begin{itemize}
                \item kill, get name/ID
            \end{itemize}
            \item holen von Daten
            \item [] \begin{itemize}
                \item Zeit
                \item Bildschirmgröße
                \item Tastatureingabe
            \end{itemize}
            \item Prints
            \item [] \begin{itemize}
                \item Bildschirm
                \item seriell (kprint)
                \item kerneldaten (VMAs, Apps, threads)
            \end{itemize}
            \item Pixel/Bildzeichnen
            \item Musik abspielen
        \end{itemize}

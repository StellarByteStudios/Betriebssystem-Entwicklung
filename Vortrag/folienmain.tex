% !TeX TXS-program:compile = txs:///pdflatex/[--shell-escape]

\pdfoptionpdfminorversion=5
\documentclass[27pt, aspectratio=169, t ,hyperref={pdfpagelabels=false}]{beamer}
\usepackage{pgfpages}
%\setbeameroption{show notes on second screen=right}



\mode<presentation> {
    \usetheme{HHUD}
    \setbeamercovered{invisible}
}
\usepackage[ngerman]{babel}
\usepackage[utf8]{inputenc}
\usepackage{times}
\usepackage[T1]{fontenc}
\usepackage{amsmath}
\usepackage{subfigure}
\usepackage{graphicx}
\usepackage{hyperref}
\usepackage{xmpmulti}
\usepackage{multicol}
\usepackage{icomma}
\usepackage{csquotes}
\usepackage{listings}
%\usepackage{minted} % Funktioniert grade noch nicht 
\usepackage[backend=biber,style=numeric,url=false]{biblatex}

\bibliography{references.bib}
% If you want to exclude the appendix from the frame counter you have to use the appendixnumberbeamer package. But be aware that the current version causes a problem with the frame counter.
\usepackage{appendixnumberbeamer}

%% Die folgenden Zeilen können auskommentiert werden, um vor jedem Kapitel eine Gliederungsfolie einzufügen
% \AtBeginSection[] {
%   \begin{frame}<beamer>
%     \thispagestyle{empty}
%     \frametitle{Gliederung}
%     \vspace{-5mm}
%     \tableofcontents[currentsection]
%   \end{frame}
% }

\usebackgroundtemplate{\includegraphics[width=\paperwidth]{fig/Vorlagen/background_cd_2020}}

\newcommand{\backgroundNormal}{\usebackgroundtemplate{
    \includegraphics[width=\paperwidth]{fig/Vorlagen/background_cd_2020}}}
\newcommand{\backgroundTitle}{\usebackgroundtemplate{
    \includegraphics[width=\paperwidth]{fig/Vorlagen/background_heine}}}
\newcommand{\backgroundEmpty}{\usebackgroundtemplate{
    \includegraphics[width=\paperwidth]{fig/Vorlagen/background_empty}}}
    
\setlength{\leftmargini}{9pt}
\setbeamersize{text margin left=25pt,text margin right=25pt} 
\setbeamertemplate{itemize/enumerate subbody end}{\vspace{.5\baselineskip}}

% =============== Meta-Daten =============== %
\title{Betriessystementwicklung:\\Shell-Implementierung und plattformunabhängige Apps}
\author{Carsten Krollmann}
\date{9.7.2025}
\institute{Institut für Informatik\\Heinrich-Heine-Universität Düsseldorf}
\subject{Informatik}

%
% Hier beginnt das Dokument
%


% =============== Eigene Präambel-Befehle =============== %
% Kopieren von Folien, wenn nur Bilder sich ändern
\usepackage{clipboard}


% Darstellung von Pseudocode
\usepackage{algorithm}
\usepackage{algpseudocode} 

% Darstellung der Tabelle
\usepackage{booktabs}

% Footnotes auf Seite ganz nach unten schieben
\usepackage[bottom]{footmisc}

% anzeigen von Unicode
\usepackage{listings}
\lstset{
    basicstyle=\ttfamily\footnotesize,
    columns=fullflexible,
    breaklines=true,
    tabsize=4
}


% Graphendarstellung
\usepackage{tikz}
\usetikzlibrary{intersections, 
  arrows.meta, 
  positioning, 
  shapes.geometric, 
  decorations.pathmorphing,
  calc
}

\usepackage{enumitem}% bessere aufzählungen
% Standartaufzählung wieder verwenden
\setlist[itemize,1]{label=\textbullet}
\setlist[itemize,2]{label=--}
\setlist[itemize,3]{label=$\ast$}


% Schriftgröße der Zitierungen
% fontsize{x}{y} gibt x pt Schriftgröße und y pt Zeilenabstand
\renewcommand{\footcite}[1]{%
  \footnote{%
    \fontsize{6}{8}\selectfont%
    \fullcite{#1}%
  }%
}

% Counter gegen Footnote duplicates
\newcounter{fnnumber} 


% Gedankenblasen-Makro
\newcommand{\gedankenblase}[6]{%
  \begin{scope}
    \small
    % Hauptblase
    \draw[fill=#4!60, draw=#5, thick] (#1,#2) ellipse (2cm and 1cm);
    \node at (#1,#2) {\parbox{3cm}{\centering #3}};
    % Punkte nach unten
    \draw[fill=#4!50, draw=#5] (#1+1*#6,#2-1.15) circle (0.2cm);
    \draw[fill=#4!30, draw=#5] (#1+1.3*#6,#2-1.5) circle (0.1cm);
  \end{scope}
}


% ===============    =============== %



\begin{document}

\backgroundTitle
  \begin{frame}
    \thispagestyle{empty}
    \begin{columns}
    \column{0.4\paperwidth}
    {
    \footnotesize
    \color{hhuBlau}
    \put(20,-200){\insertdate}
    
    }
    \column{0.6\paperwidth}
    
    \color{hhuBlau}
    \LARGE \inserttitle\\[\baselineskip]
    
    \large \insertauthor
    \end{columns}
  \end{frame}
  \backgroundNormal


  \begin{frame}
    \thispagestyle{empty}
    \frametitle{Gliederung}
    \vspace{-5mm}
    \tableofcontents
  \end{frame}

  % % % % % % % % % % Ab hier werden die LaTeX-Dateien der einzelnen Abschnitte eingefügt % % % % % % % % % %

  \section{Einleitung - Ziele und Motivation}


\begin{frame}{Motivation}
    \begin{Large}
        Apps sind auf Betriebssystem angewiesen \newline
        $\Rightarrow$ starke Abhängigkeit \newline
        \newline
        \onslide<2>
        Apps auf unterschiedlichen Systemen \newline
        $\Rightarrow$ enge Schnittstelle \newline
    \end{Large}
\end{frame}


\begin{frame}{Ziele}
    \begin{Large}
        Ziele:
    \end{Large}
    \vspace{15pt}

    \begin{itemize}
        \item Schnittstelle als Lib
        \item Kernel implementiert Schnittstelle
        \item Apps über Schnittstelle
        \item [] \quad $\Rightarrow$ Systeme austauschbar
    \end{itemize}
\end{frame}


  \section{Userlib Schnittstelle}



\begin{frame}{Userlib}
    \begin{Large}
        Userlib als Grundlage für Schnittstelle
    \end{Large}
    \vspace{15pt}

    \begin{itemize}
        \item Kapselung des Kernels
        \item Apps nur gegen Userlib
        \item Verbindung zu Kernel via Syscalls
    \end{itemize}
    
\end{frame}


\begin{frame}{Userlib}
    \begin{Large}
        $\Rightarrow$ Beide Kernel implementieren Userlib \newline \newline
        $\Rightarrow$ Apps laufen unabhänging
    \end{Large}
    \vspace{15pt}
\end{frame}


\begin{frame}{Userlib Erweiterung}
    \begin{Large}
        Erweiterung der lib: \cite{usrlib-repo}
    \end{Large}
    \vspace{15pt}

    \begin{itemize}
        \item mehr Syscalls
        \item wrapper um die Syscalls
        \item Musik
        \item Bilder
        \item System Management
    \end{itemize}
\end{frame}


\begin{frame}{Userlib Aufbau}
    \begin{Large}
        Aufbau \cite{usrlib-repo}
    \end{Large}
    \vspace{15pt}

    \begin{minipage}[t]{0.48\textwidth}
        \verb|├── graphix
    │   ├── gprint.rs
    │   ├── picturepainting
    │   │   ├── animate.rs
    │   │   ├── paint.rs
    │   │   └── pictures
    │   │       ├── blinking
    │   │       │   ├── blinking0.rs
    │   │       │   ├── ...
    │   │       │   └── blinking4.rs
    │   │       └── charmander
    │   │           ├── charmander0.rs
    │   │           ├── ...
    │   │           └── charmander4.rs
    │   └── print_setpos.rs
    ├── music
    │   ├── note.rs
    │   └── player.rs
    ├── time
    │   └── rtc_date_time.rs|
    \end{minipage}
    \hfill
    \begin{minipage}[t]{0.48\textwidth}
        \verb|├── kernel
    │   ├── allocator
    │   │   ├── allocator.rs
    │   │   ├── listnode.rs
    │   │   └── list.rs
    │   ├── kprint.rs
    │   ├── runtime
    │   │   ├── environment.rs
    │   │   ├── env_variables.rs
    │   │   └── runtime.rs
    │   ├── shell
    │   │   └── command_parser.rs
    │   └── syscall
    │       ├── process_management.rs
    │       └── user_api.rs
    ├── lib.rs
    ├── music
    │   ├── note.rs
    │   └── player.rs
    └── utility
        ├── delay.rs
        ├── mathadditions
        │   ├── complex_numbers.rs
        │   ├── fibonacci.rs
        │   └── math.rs
        └── queue.rs|
    \end{minipage}
\end{frame}



\begin{frame}{Userlib Aufbau}
    \begin{minipage}[t]{0.3\textwidth}
        \includegraphics[height=0.8\textheight]{fig/Code Screens/Enum1.png} 
    \end{minipage}
    \hfill
    \begin{minipage}[t]{0.65\textwidth}
        \includegraphics[height=0.8\textheight]{fig/Code Screens/Enum2.png} 
    \end{minipage}
\end{frame}

  \section{Shellfunktion}


\begin{frame}{Shell}
    \begin{Large}
        Shell läuft im Kernel
    \end{Large}
    \vspace{15pt}

    \begin{itemize}
        \item Liest Tastatur aus
        \item setzt Befehle zusammen
    \end{itemize}
    
\end{frame}


\begin{frame}{Shell}
    \begin{Large}
        Shellfunktionen: 
    \end{Large}
    \vspace{15pt} 
    
    \begin{itemize}
        \item Erfassen der Eingabe
        \item parsen der Befehle
        \item [] \begin{itemize}
            \item Argumente zwerpflücken
            \item Environmentvariabeln
        \end{itemize}
        \item starten der Apps
        \item [] \begin{itemize}
            \item App wird geladen
            \item Scheduler richtet Prozess ein
            \item Neuer Thread (Mapping + Environment)
        \end{itemize}
    \end{itemize}
\end{frame}



  \section{Apps}



\begin{frame}{Apps Liste}
    \begin{Large}
        Folgende Apps wurden implementiert
    \end{Large}
    \vspace{15pt}

    \begin{itemize}
        \item animation
        \item apps
        \item datetime
        \item echo
        \item kill
        \item mandelbrot
        \item music
        \item pic
        \item threads
        \item uptime
    \end{itemize}
    
\end{frame}



  \section{Demo}

\begin{frame}{Demo Apps}
    \begin{Large}
        Folgende Apps werden vorgeführt
    \end{Large}
    \vspace{15pt}

    \begin{itemize}
        \item echo (Environment-Variablen)
        \item animation (visuell)
        \item turtel (eingeben)
        \item music (audio)
        \item threads (management)
        \item kill (management)
    \end{itemize}
    
\end{frame}


\begin{frame}{Demo App}
    \begin{Large}
        Laufen Apps jetzt auch unabhängig?
    \end{Large}
    \vspace{15pt} 
    
    $\Rightarrow$ Zeige an music app
\end{frame}




  
  %\printbibliography

  
  \appendix
  % anhang mit Quellenverzeichniss
  \section{Anhang}

\begin{frame}[allowframebreaks]{Quellenangaben}
    \printbibliography
\end{frame}




  

  % % % % % % % % % % Ende der eingefügten LaTeX-Dateien % % % % % % % % % %

\end{document}

%
% Hier endet das Dokument
%

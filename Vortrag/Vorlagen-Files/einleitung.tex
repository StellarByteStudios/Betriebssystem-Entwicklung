\section{Einleitung}

\subsection{Foliengestaltung}

\begin{frame}{Tipps für die Präsentation}
\begin{itemize}
\item Keine vollständigen Sätze
\item Folien nicht überladen (präsentieren, nicht ablesen!)\footnote{diese Präsentation ist ein schlechtes Beispiel}
\item max. 1 Folie pro Minute Vortragszeit (ansonsten hat man Folien, die keinen Mehrwert für die Präsentation haben)
\item lieber etwas zu schnell fertig sein statt zu überziehen
\begin{itemize}
    \item BA: max. 20 Minuten + 10 Minuten Fragen
    \item MA: max. 30 Minuten + 15 Minuten Fragen
\end{itemize}
\item Einleitung: Motivation und Ziel der Arbeit schnell klarmachen
\begin{itemize}
    \item[$\rightarrow$] davon ausgehen, dass Zuhörer nicht schlauer sind als du vor Beginn der Arbeit
\end{itemize}
\item grobe Struktur (je nach Thema): Ausgangssituation (was war schon da?), Ziel (was soll besser werden?), das hab ich gemacht, das ist rausgekommen
\item bei Anwender-Software: Screenshot nicht vergessen (Live-Demos gehen meistens schief)
\end{itemize}
\end{frame}

\begin{frame}{Einstieg in die Präsentation}
    \begin{itemize}
        \item während das Titel-Slide angezeigt: in Thema einführen \& klarmachen, \emph{warum} man \emph{was} getan hat
        \item Einstieg ca. $1$–$1,5$ Minuten, Empfehlung:
        \begin{enumerate}
            \item Opening Statement: Einführung in Thema, z.\,B. rhetorische Frage, oder ABT-Schema (vgl. nächstes Slides)
            \item Begrüßung
            \item Vorstellung
            \begin{itemize}
                \item auch wenn man schon vorgestellt wurde und damit Aussprache klar
                \item bei BA/MA reicht die Nennung des Namens, da klar ist, wo und was man studiert
            \end{itemize}
            \item Titel
            \begin{itemize}
                \item damit z.\,B. Aussprache von Programmnamen klar
            \end{itemize}
            \item ggf. Danksagungen
            \begin{itemize}
                \item relevant bei geförderten Forschungsprojekten, eher nicht bei BA/MA)
            \end{itemize}
            \item Was kann das Publikum aus dem Vortrag mitnehmen? (The audience is the star of the show - not you!)
            \begin{itemize}
                \item z.\,B. mit you-Messages arbeiten
                \item Übersicht über Hauptthemen, die kommen werden
            \end{itemize}
        \end{enumerate}
    \end{itemize}
\end{frame}

\begin{frame}{Beispiel: ABT opening}
    And-But-Therefore:
    \begin{itemize}
        \item Handlungsrahmen mit zwei Sätzen aufziehen
        \item offene Frage aufzeigen
        \item Forschungsfrage herleiten
    \end{itemize}
\small
    \begin{quotation}
Many people exchange arguments, for exmaple on political issues, on the Internet, \underline{and} you might ask yourself: How similar are my attitudes compared to those persons on the Internet? \underline{But} there are no sensible methods to determine the similarity of argumentations. \underline{Therefore}, we have developed a metric for calculating the distance between 2 argumentation graphs.

Hello, my name is Ada Lovelace, and I work in the arugmentation research group at the Heinrich Heine University in Düsseldorf in Germany. Today, I'd like to present my work “Fancy Title Of Work.”

Within the next 20 minutes, I'll explain why it can be useful for you to compare attitudes in argumentations, how you can do that, and how we made sure that our method yields intuitive results.

Questions can be asked in the end, and now let's start with an exmaple.
    \end{quotation}
\end{frame}

\begin{frame}
  \frametitle{Blöcke}

  Sehr praktisch ist die Verwendung von Blöcken:
  \begin{block}{Normaler Blocktitel}
    Blöcke sind zur Hervorhebung gedacht. In normalen Blöcken können
    wichtige Erkenntnisse (Zwischenergebnisse) stehen, die nicht
    unbemerkt bleiben dürfen.
  \end{block}

  \begin{exampleblock}{Beispiel-Blocktitel}
    Diese Blöcke sind für Beispiele o.ä. gedacht.
  \end{exampleblock}

  \begin{alertblock}{Alarm-Blocktitel}
    Diese Blöcke beinhalten für gewöhnlich Problembeschreibungen.
  \end{alertblock}
\end{frame}

\begin{frame}
  \frametitle{Weitere Hilfen}

  Es gibt viele Quellen, die für \LaTeX Beamer herangezogen werden können. Besonders gut sind:
  \begin{itemize}
    \item Der Beamer User Guide von Till Tantau:\\\texttt{http://www.math.binghamton.edu/erik/beameruserguide.pdf}
    \item Ein sehr gutes Beamer Tutorial von Ki-Joo Kim:\\\texttt{http://saikat.guha.cc/ref/beamer\_guide.pdf}
    \item Der jeweilige Betreuer der Abschlussarbeit :-)
  \end{itemize}

\end{frame}

\subsection{subsections bitte meiden, sie werden nur in Gliederungen angezeigt.}
